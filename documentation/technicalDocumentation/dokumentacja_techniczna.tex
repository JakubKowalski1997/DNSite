\documentclass[11pt]{article}
%%% Preambuła %%%
\usepackage[T1]{fontenc}
\usepackage[polish]{babel}
\usepackage[utf8]{inputenc}
\usepackage{lmodern}
\usepackage{hyperref}
\usepackage{mathptmx}
\usepackage{float}
\usepackage{graphicx}
\usepackage{xcolor}
\selectlanguage{polish}
\usepackage{titlesec}

\definecolor{backgroundColor}{HTML}{3D3D3D}

%%% Strona tytułowa %%%
\title 
{	
	{
		\textbf{\textsf{\Huge\color{orange}DNS\color{white}ite}} \\ [0.1in]
		\normalfont\sffamily\LARGE\color{white}
		Aplikacja webowa do zarządzania serwerem DNS \\[0.1in]
		Dokumentacja techniczna\\ [1.5in]
		\large 
		Inżynieria Oprogramowania \\
		Wydział Fizyki i Informatyki Stosowanej \\
		Informatyka Stosowana, 3 rok \\
	}
}

\author 
{
	\color{white}\normalfont\sffamily Arkadiusz Kasprzak \and 
	\color{white}\normalfont\sffamily Jarosław Cierpich \and 
	\color{white}\normalfont\sffamily Jakub Kowalski \and 
	\color{white}\normalfont\sffamily Konrad Pasik \and 
	\color{white}\normalfont\sffamily Krystian Molenda
}
\date{}


\definecolor{ao}{rgb}{0.0, 0.0, 1.0}	
\definecolor{forestgreen(web)}{rgb}{0.13, 0.55, 0.13}
\definecolor{darkbrown}{rgb}{0.4, 0.26, 0.13}
\definecolor{darkorange}{rgb}{0.91, 0.41, 0.17}

\titleformat{\section}
  {\normalfont\sffamily\Large\bfseries\color{darkorange}}
  {\thesection}{1em}{}

\titleformat{\subsection}
  {\normalfont\sffamily\large\bfseries\color{darkorange}}
  {\thesubsection}{1em}{}

\titleformat{\subsubsection}
  {\normalfont\sffamily\normalsize\bfseries\color{darkorange}}
  {\thesubsubsection}{1em}{}
	

\begin{document}


%%% Strona tytułowa %%%
\pagecolor{backgroundColor}
\maketitle
\thispagestyle{empty}


\newpage
\clearpage
\pagenumbering{arabic}
\pagecolor{white}

%%% Spis treści %%%
\tableofcontents

\newpage 

\section{Wstęp}
\textbf{DNSite} to aplikacja webowa do zarządzania serwerem DNS. Dostarcza ona użytkownikowi możliwości łatwej i wygodnej edycji danych związanych z serwerem PowerDNS przechowywanych w bazie PostgreSQL.\newline
Niniejsza dokumentacja techniczna została przygotowana dla pierwszego pełnego wydania aplikacji. Zawiera informacje przydatne przy dalszym rozwoju aplikacji.

\section{Budowanie aplikacji}

\section{Architektura aplikacji}

\section{Stos technologiczny}

\section{Warstwa frontend}

\section{Warstwa backend}

\section{Testy}

\section{Lista możliwych rozszerzeń i poprawek}






\end{document}